\documentclass{jlreq}

\usepackage{amsmath, amssymb}
\usepackage{enumerate}
\usepackage{tikz}
\usepackage{listings, xcolor}

\lstset{
  basicstyle = {\ttfamily}, % 基本的なフォントスタイル 
  frame = {tbrl}, % 枠線の枠線。t: top, b: bottom, r: right, l: left
  breaklines = true, % 長い行の改行
  numbers = left, % 行番号の表示。left, right, none
  showspaces = false, % スペースの表示
  showstringspaces = false, % 文字列中のスペースの表示
  showtabs = false, % タブの表示
  keywordstyle = \color{blue}, % キーワードのスタイル。intやwhileなど
  commentstyle = {\color[HTML]{1AB91A}}, % コメントのスタイル
  identifierstyle = \color{black}, % 識別子のスタイル 関数名や変数名
  stringstyle = \color{brown}, % 文字列のスタイル
  captionpos = t % キャプションの位置 t: 上、b: 下
}

\title{情報学群実験第2 第1回レポート}
\author{学籍番号:1280391 細川 夏風}
\date{\today}

\begin{document}

  \maketitle
  \section{はじめに}
  機械学習(Machine Lerning:ML)とは,参考文献\cite{ml}より与えられたデータから主として統計処理によって有用な情報を抽出することを指す.
  \section{実験内容}
  私は,第4,5回に行った「物体認識」について述べる.この物体認識の技術の多くはそこになにかが存在しているということを判別するために用いられる.例として挙げるのであれば,自動運転の際の歩行者や他の車の検知などである.更に応用として任意の線や範囲に入っている物体を検知することも可能である.これは入ってきた物体や出ていく物体などを検知するという使い方もできる.今回の実験では,YOLOという映像認識モデルを用いた.このモデルにはいくつの手順が存在する.
  \subsection{手順}
  \begin{enumerate}[1. ]
    \item 物件検出(Detection):画像の中に学習した物件を見つけ出し,周りにバウンティボックスを画く手順.認識できるものをとその位置を見つけ出すことが可能である.
    \item 画面分割(Segmentation):画像を複数の領域に分割し,各領域が何位を表しているのかを識別する手順.このとき主にピクセルレベルで物体を特定する.それぞれのピクセルがどの物件に属するかを分類する.
    \item 姿勢推定(Pose estimation):画像や動画内の人物や物体の姿勢を推定する技術,具体的には,人体の関節や物体の重要なポイント(キーポイント)を検出し,それらの位置関係を元に,姿勢鮎誤記をするためのモデルである.
    \item 向き付きバウンティボックス(OBB:Oriented Bounding Box):物体検出や画像処理において,物体を囲むために使用されるボックスが,物体の向きに合わせて回転したバージョンのバウンティボックスのこと.通常のバウンティボックスは縦軸と横軸に対して並行なのに対して,OBBは余分な領域を最小限に抑えることができるため,より正確に境界を表現できる.:w
  \end{enumerate}
  
  \section{考察}
  この実験の課題は動画中のアリの判別を行うというものであった.しかし,私はなかなかアリの判別を行うことができなかった.常に$1$体かもしくは$0$体のみしか判別していなかった.隣の友人もそのような症状に見舞われていた.おそらく多くの人がそのような症状に見舞われていただろう.理由はアルゴリズムやプログラムコードのあったわけではなかった.至って単純に用いたデータに問題があった.用いたデータは前年度の先輩たちが残したものである.ある時間において,そのとき写っていたアリの数をラベルとしてファイルに収めていたものを利用していた.このデータの中身をよく見ていると,しっかりとそのアリの数や時間が記されているものはほとんど無く,いわば``ゴミデータ''であった.これにより,本来はその時間にはいるはずのアリがラベルが存在しないため,存在しないという事になってしまった.YOLOにいくらとてつもない精度があったとしても,YOLOは教師あり学習という方法を用いているため,その教師データが正確でなければ精度は著しく低下する.そのようなことが今実験では理解できた.これはデータを用いる中でとても大事な考え方``GIGO:Garbage In, Garbage Out''である.質の悪いデータを用いれば,質の悪い結果しか得られないということである.もし私がこのような技術をを用いて,何かしらのシステムを作成する際は,データをきれいに保つということを心がけて作成する必要があるということがわかった.また,私はこの実験で用いられたモジュール等の本来のアルゴリズムの一切を理解していないというのも大きな問題であると感じた.近年ではAIが多くの分野で活躍しているため,その根幹的な技術の一端程は理解しておくべきだと感がじた.
  \section{まとめ}
  今回の実験を通して,GIGOの重要性とAI技術の根幹的な理解の重要性を学ぶことができた.このような技術を活かす場として考えられるのは,やはり店舗の入りを見るという方法だろう.これによって店の前に何人の人がいて,どれくらいの人が入店するのかを予測することが可能になるかもしれない.また,災害時のデータを収集すれば,災害発生時に落石や土砂崩れなどがどのくらい指定の範囲に新入試てくるかを予測するということも可能になるだろう.
\begin{thebibliography}{99}
  \bibitem{ml} 著者:中川 裕志 編集:東京大学高額教程委員会, 機械学習, 丸善出版 平成27年11月20日.
  \bibitem{ilusto_ml} 著者:杉山 将,イラストで学ぶ 機械学習,講談社,2014年9月10日.
  \bibitem{AI_jiten} 著者;編者 中島 秀之,浅田 稔,橋田 浩一,松原 仁,山川 宏,栗原 聡,松尾 豊,近代科学社,2021年1月31日
\end{thebibliography}
\end{document}
