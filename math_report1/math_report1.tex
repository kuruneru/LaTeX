\documentclass[uplatex]{jsarticle}

\usepackage{amssymb}

\title{離散数学 演習課題レポート1}
\author{細川 夏風}
\date{\today}

\begin{document}

\maketitle

\newpage

\section{問1}
\renewcommand{\theenumi}{(\arabic{enumi})}
\begin{enumerate}
\vspace{12pt}

\item 
	\begin{enumerate}
		\item 吉原秀くんのGPAは3.0以上である.
		\item $\sqrt{2}$は無理数である.
		\item 日本語は英語ではない.
	\end{enumerate}

\vspace{12pt}

\item
	\begin{enumerate}
		\item ''アルジャーノンに花束を''という小説はとても面白い.
		\item 藤井聡太7冠はとてもすごい.
		\item ある学生はとても偉い.
	\end{enumerate}
\vspace{12pt}

\item
	\begin{enumerate}
		\item ある実数xは0以上の整数である.
		\item ある情報学群の学生Aはドミトリーに住んでいる.
		\item ある国Bは経度135°の位置に存在している.
	\end{enumerate}

\end{enumerate}

\section{問2}
\renewcommand{\theenumi}{(\arabic{enumi})}
\begin{enumerate}
\vspace{12pt}

\item
	\begin{enumerate}
		\item すべての情報学群の学生は情報代数の授業を履修している.
		\item すべての車好きは頭文字Dを閲覧したことがある.
		\item すべての飲食店は禁煙席が存在する.
	\end{enumerate}

\vspace{12pt}
\item	
	\begin{enumerate}
		\item ある情報学群の学生は情報代数の授業を履修していない.
		\item ある車好きは頭文字Dを閲覧したことがない.
		\item ある飲食店は禁煙席が存在しない.
	\end{enumerate}

\vspace{12pt}
\item
	\begin{enumerate}
		\item ある情報学群の学生は筋トレをしていない.
		\item ある政治家は汚職をしている.
		\item ある情報学群の先生はコーラを持参している.
	\end{enumerate}

\vspace{12pt}
\item
	\begin{enumerate}
		\item すべての情報学群の学生は筋トレをしている.
		\item すべての政治家は汚職をしていない.
		\item すべての情報学群の先生はコーラを持参していない.
	\end{enumerate}
\end{enumerate}

\section{問3}
\begin{enumerate}
\vspace{12pt}

\item
	\begin{enumerate}
		\item すべての情報学群の学生が履修している科目が存在する.
		\item どんなコンピュータにもインストールされているソフトウェアが存在する.
		\item ある人間はすべての大学生に嫌われている.
	\end{enumerate}
\vspace{12pt}
\item
	\begin{enumerate}
		\item すべての科目について、その科目を履修していない学生が存在する.
		\item すべてのソフトウェアについて、そのソフトウェアがインストールされていないコンピュータが存在する.
		\item すべての人間はある大学生に嫌われていない.
	\end{enumerate}
\item $x + y = y + x$を満たすような任意の自然数$x$とある自然数$y$が存在する.
\end{enumerate}

\section{問4}
\begin{enumerate}
\vspace{12pt}

	\item $\exists x ( 2x = n ), x \in \mathbb{Z}_+$
	\item 上記の命題について、$n = 28$のとき、論理式$\exists x ( 2x = 28 )$となる.$x = 14$のとき、$28 = 2 \times 14 = 2x $が成り立つためこの命題は真である.
	\item $A \subset B \leftrightarrow \forall x (x \in A \rightarrow x \in B)$
	\item 命題について集合$F$、$T$について、$F = \{x | \exists k(x = 4k)\}$.ただしこのとき、$k, x \in \mathbb{Z}$である.$T = \{ y | \exists l (x=2l)\}$.ただし、このとき、$l,y \in \mathbb{Z}$である.このとき、$a' \in F$がを満たすような$a'$を任意にとる.このとき$a'=4k$を満たすような整数$k'$をとれる.$l' =2k'$とおく、$a' = 4k' =2l'$が成り立つため、$a' \in T$となる以上より、$F \subset T$となる.
	\item 
			否定命題 $\exists x (x \in P \land x \not\in E )$    $(x \in \mathbb{Z}_+)$を示す.
			$x = 2$のとき、$2 \in P$かつ$2 \not\in E$である.よって、$x \in P \land x \not\in E$を満たす$x$が存在するため、否定命題$\exists x ( x \in P \land x \not\in E )$は成り立つ.よって順命題は偽である.

\end{enumerate}

\section{問5}
\begin{enumerate}
	\item 命題:任意の整数$n$について、$n^2 + 1$は偶数である.
		\begin{enumerate}
			\item この命題を論理式で表すと$\forall n\exists k (n^2 + 1 = 2k)$と表すことができる.このとき$k,n \in \mathbb{Z}$.この否定命題$\exists n \forall k (n^2 + 1 \neq 2k)$について考える.$2k$について、任意の2の倍数となる$k'$とれる.このとき、$\exists n (n^2 + 1 \neq k')$は$n = 2$のとき、この否定命題が成り立つため、順命題は成り立たない.
		\end{enumerate}
\end{enumerate}

\end{document}