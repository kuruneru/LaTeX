\documentclass[unicode,11pt,aspectratio=169]{beamer}

% 2. 日本語設定 (LuaTeX-ja)
\usepackage{luatexja}
\usepackage{luatexja-preset} % フォントはお好みで変更してください

% 3. テーマ設定 (Madridはシンプルで使いやすい)
\usetheme{Madrid}
\usecolortheme{seahorse} % 配色を少し柔らかく

% 4. 便利なパッケージ
\usepackage{bookmark} % 索引・しおり用
\usepackage{graphicx} % 画像用
\usepackage{listings} % コード表示用
\usepackage{xcolor}   % 色用

% 5. コードブロックの設定 (プログラミング用)
\lstset{
    basicstyle=\ttfamily\small,
    keywordstyle=\color{blue},
    commentstyle=\color{gray},
    stringstyle=\color{red},
    frame=single,
    breaklines=true
}

% --- 表紙の情報 ---
\title{TCP/IP輪講 第一回}
\subtitle{サブタイトルをここに記述}
\author{細川夏風}
\institute{高知工科大学 情報学群}
\date{\today}

\begin{document}

% タイトルスライド
\begin{frame}
    \titlepage
\end{frame}

% 目次スライド
\begin{frame}{目次}
    \tableofcontents
\end{frame}

\section{はじめに}

% 通常のスライド
\begin{frame}{スライドの構成}
    \begin{itemize}
        \item 箇条書きの1番目
        \item 箇条書きの2番目
        \begin{itemize}
            \item ネストした項目
        \end{itemize}
        \item 数式の記述も可能です:$E = mc^2$
    \end{itemize}
\end{frame}

\section{技術的な内容}

% 2段組みのスライド
\begin{frame}{2段組みの例}
    \begin{columns}
        \begin{column}{0.5\textwidth}
            左側のコンテンツ
            \begin{itemize}
                \item 説明文など
                \item テキスト情報
            \end{itemize}
        \end{column}
        \begin{column}{0.5\textwidth}
            %
            右側のコンテンツ
            \begin{block}{重要なポイント}
                ここに強調したい内容を書きます。
            \end{block}
        \end{column}
    \end{columns}
\end{frame}

% コードスライド (fragileが必要)
\begin{frame}[fragile]{コードの例}
\begin{lstlisting}[language=C]
#include <stdio.h>
int main() {
    printf("Hello, Beamer!\n");
    return 0;
}
\end{lstlisting}
\end{frame}

\end{document}

