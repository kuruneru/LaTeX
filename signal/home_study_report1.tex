\documentclass{jlreq}

\usepackage{amsmath, amssymb}
\usepackage{enumerate}
\usepackage{tikz}
\usepackage{listings, xcolor}
\usepackage{paracol}

\lstset{
  basicstyle = {\ttfamily}, % 基本的なフォントスタイル 
  frame = {tbrl}, % 枠線の枠線。t: top, b: bottom, r: right, l: left
  breaklines = true, % 長い行の改行
  numbers = left, % 行番号の表示。left, right, none
  showspaces = false, % スペースの表示
  showstringspaces = false, % 文字列中のスペースの表示
  showtabs = false, % タブの表示
  keywordstyle = \color{blue}, % キーワードのスタイル。intやwhileなど
  commentstyle = {\color[HTML]{1AB91A}}, % コメントのスタイル
  identifierstyle = \color{black}, % 識別子のスタイル 関数名や変数名
  stringstyle = \color{brown}, % 文字列のスタイル
  captionpos = t % キャプションの位置 t: 上、b: 下
}

\title{信号理論基礎 課題レポート1}
\author{細川 夏風}
\date{\today}

\begin{document}

  \maketitle

  授業より,$a_m = \frac{1}{\int_{0}^{4}(f_m)^2dt}\int_{0}^{4}ff_{m}dt$,
  $ f_1 = 
    \begin{bmatrix}
      1 \\ 1 \\ 1 \\ 1
    \end{bmatrix}
  $,
  $ f_2 = 
    \begin{bmatrix}
      1 \\ 1 \\ -1 \\ -1
    \end{bmatrix}
  $,
  $ f_3 = 
    \begin{bmatrix}
      1 \\ -1 \\ -1 \\ 1
    \end{bmatrix}
  $,
  $ f_4 = 
    \begin{bmatrix}
      1 \\ -1 \\ 1 \\ -1
    \end{bmatrix}
  $,を用いる.

    \begin{enumerate}[(1). ]
      \item 
        $ f = 
          \begin{bmatrix}
            5 \\ 1 \\ -5 \\ 3
          \end{bmatrix}
        $
        のときの$a_1$\~{}$a_4$を求めよ.
      
        \vspace{18pt}

        \begin{paracol}{2}
          \begin{flushleft}
            $
            \begin{aligned}
              a_1 &= \frac{1}{4}\int_{0}^{4}ff_1dt \\
              &= \frac{1}{4} \langle f, f_1 \rangle \\
              &= \frac{1}{4}\left\{ 5 \cdot 1 + 1 \cdot 1 + (-5) \cdot 1 + 3 \cdot 1 \right\} \\
              &= \frac{1}{4} \cdot 4 \\
              &= 1
            \end{aligned}
            $

            よって$a_1 = 1$.

            \vspace{18pt}

          $
          \begin{aligned}
            a_2 &= \frac{1}{4}\int_{0}^{4}ff_2dt \\
            &= \frac{1}{4} \langle f, f_2 \rangle \\
            &= \frac{1}{4}\left\{ 5 \cdot 1 + 1 \cdot 1 + (-5) \cdot (-1) + 3 \cdot (-1) \right\} \\
            &= \frac{1}{4} \cdot 8 \\
            &= 2
          \end{aligned}
          $よって$a_2 = 2$.  
        \end{flushleft}

        \vspace{18pt}

        \switchcolumn

        \begin{flushleft}
          $
          \begin{aligned}
            a_3 &= \frac{1}{4}\int_{0}^{4}ff_3dt \\
            &= \frac{1}{4} \langle f, f_3 \rangle \\
            &= \frac{1}{4}\left\{ 5 \cdot 1 + 1 \cdot (-1) + (-5) \cdot (-1) + 3 \cdot 1 \right\} \\
            &= \frac{1}{4} \cdot 12 \\
            &= 3
          \end{aligned}
          $
          よって$a_3 = 3$.

          \vspace{18pt}

          $
          \begin{aligned}
            a_4 &= \frac{1}{4}\int_{0}^{4}ff_4dt \\
            &= \frac{1}{4} \langle f, f_4 \rangle \\
            &= \frac{1}{4}\left\{ 5 \cdot 1 + 1 \cdot (-1) + (-5) \cdot 1 + 3 \cdot (-1) \right\} \\
            &= \frac{1}{4} \cdot (-4)\\
            &= -1
          \end{aligned}
          $
          よって$a_4 = -1$.
      \end{flushleft}
    \end{paracol}

    \newpage

    \item
    $ f = 
      \begin{bmatrix}
        4 \\ 0 \\ 0 \\ 0
      \end{bmatrix}
    $
    のときの$a_1$\~{}$a_4$を求めよ.
    \begin{paracol}{2}
      \begin{flushleft}
        $
        \begin{aligned}
          a_1 &= \frac{1}{4}\int_{0}^{4}ff_1dt \\
          &= \frac{1}{4} \langle f, f_1 \rangle \\
          &= \frac{1}{4}\left\{ 4 \cdot 1 + 0 \cdot 1 + 0 \cdot 1 + 0 \cdot 1 \right\} \\
          &= \frac{1}{4} \cdot 4 \\
          &= 1
        \end{aligned}
        $

        よって$a_1 = 1$.

        \vspace{18pt}

        $
        \begin{aligned}
          a_2 &= \frac{1}{4}\int_{0}^{4}ff_2dt \\
          &= \frac{1}{4} \langle f, f_2 \rangle \\
          &= \frac{1}{4}\left\{ 4 \cdot 1 + 0 \cdot 1 + 0 \cdot (-1) + 0 \cdot (-1) \right\} \\
          &= \frac{1}{4} \cdot 4 \\
          &= 1
        \end{aligned}
        $よって$a_2 = 1$.  
      \end{flushleft}

      \vspace{18pt}

      \switchcolumn

      \begin{flushleft}
        $
        \begin{aligned}
          a_3 &= \frac{1}{4}\int_{0}^{4}ff_3dt \\
          &= \frac{1}{4} \langle f, f_3 \rangle \\
          &= \frac{1}{4}\left\{ 4 \cdot 1 + 0 \cdot (-1) + 0 \cdot (-1) + 0 \cdot 1 \right\} \\
          &= \frac{1}{4} \cdot 4 \\
          &= 1
        \end{aligned}
        $
        よって$a_3 = 1$.

        \vspace{18pt}

        $
        \begin{aligned}
          a_4 &= \frac{1}{4}\int_{0}^{4}ff_4dt \\
          &= \frac{1}{4} \langle f, f_4 \rangle \\
          &= \frac{1}{4}\left\{ 4 \cdot 1 + 0 \cdot (-1) + 0 \cdot 1 + 0 \cdot (-1) \right\} \\
          &= \frac{1}{4} \cdot 4\\
          &= 1
        \end{aligned}
        $
        よって$a_4 = 1$.
      \end{flushleft}
    \end{paracol}    
  \end{enumerate}
\begin{thebibliography}{99}
  \bibitem n無し
\end{thebibliography}
\end{document}
