\documentclass[dvipdfmx]{jlreq}

	\usepackage{enumerate}
	\usepackage{amsmath}
	\usepackage{amssymb}
	\usepackage{amsfonts}

	\title{情報科学応用 第4回課題}
	\author{細川 夏風}
	\date{\today}

\begin{document}

	\maketitle

	\section{問題A}
		$m, n, k \in \mathbb{Z}$	
		\begin{enumerate}[(1)]
			\item $2 > 0 \land \exists k(-2020 = 2k)$
			\item $\forall m \exists k(0 = mk)$
			\item $\forall m (m > 0 \land \exists k(0 = mk))$
			\item $\exists n ((0 > 0) \land \exists k(n = 0k))$
			\item $\forall m \exists k(-1 = mk)$
		\end{enumerate}

	\section{問題B}
		\begin{enumerate}[(1)]
			\item この命題に対して、論理式$2 > 0 \land \exists k(-2020 = 2k)$を$2 > 0 \land$の部分と$\exists k(-2020 = 2k)$に分けて考える.$2 > 0$に付いて2は0より大きいので真.$\exists k(-2020 = 2k)$について、$k = -1010$のとき成り立つため真である.よってこの命題は真である.
			\item この命題の論理式$\forall m \exists k(0 = mk)$から考える.任意の整数となるような$m'$をとったとき、$k = 0$をとると$m' \times 0$は$0$になるためこの命題は真である.
			\item この命題の論理式$\exists m(m \leq 0 \lor \forall k(0 \neq mk))$から考える.このとき、$m = 0$であればこの命題は成り立つ.よって否定命題が真であるため順命題は偽である.
			\item この命題の論理式$\exists n ((0 > 0) \land \exists k(n = 0k))$から考える.この命題は論理式の前部分の$(0 > 0)$の部分で成り立っていないため、偽である.
			\item この命題の論理式$\forall m \exists k(-1 = mk)$の否定命題$\exists m \forall k (-1 \neq mk)$について考える.任意の整数$k'$をとる.このとき、$m = 0$とすると、$mk' \neq 0$となるため否定命題が成り立つため、順命題は偽である.
		\end{enumerate}

\end{document}

