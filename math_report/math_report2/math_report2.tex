\documentclass[uplatex]{jsarticle}

\title{離散数学 演習課題レポート2}
\author{細川 夏風}
\date{\today}

\usepackage{tikz}
\usepackage{tikz-cd}
\usepackage{amssymb}
\usepackage{amsmath}

\begin{document}

\maketitle

\section{問1}

\renewcommand{\theenumi}{(\arabic{enumi})}
\begin{enumerate}
\vspace{12pt}
	\item
		\begin{enumerate}
			\item 図1は写像ではない:このとき$f:c \mapsto 3$と定める.
			\item 図2は写像である
			\item 図3は$h:b \mapsto 1$定める.
		\end{enumerate}
	
	\item 
		以下の集合の関係について、写像であるものとそうでないものを答えなさい.また、写像でないものについては、どのように修正すれば写像の定義に沿ったものになるか答えなさい.
		\begin{enumerate}

			\item
			\begin{enumerate}
				\item 
					$A=\{a,b,c,d\}$,$B=\{2,4,6,8,10\}$について$f=\{(a,2),(b,4),(d,6),(d,8)\}$
				\item
					$C=\{a,b\}$,$D=\{1,5,7\}$について$f=\{(a,1)\}$
				\item
					$E=\{a,b,c,d\}$,$F=\{100\}$について$f=\{(a,100),(b,100),(c,100),(d,100)\}$
			\end{enumerate}
		\end{enumerate}
\end{enumerate}

\section{問2}

\begin{enumerate}
\vspace{12pt}
	\item 
		\begin{enumerate}
			\item ジュースの集合それぞれについて、その銘柄を対応付ける対応関係.
			\item ある寮の住人について、その寮の部屋に対応付ける対応関係.
		\end{enumerate}
	\item
		\begin{enumerate}
			\item (1)について定義域はジュースで値域は銘柄である.
			\item (2)について定義域は寮の住人であり、値域は寮の部屋である.
		\end{enumerate}
		
	\item $\forall x_1 \forall x_2 ( x_1 \neq x_2 \rightarrow f(x_1) \neq f(x_2))$
	\item 
		\begin{enumerate}
			\item 学生の集合を定義域、学籍番号の集合を値域としたときの学生と学籍番号の関係の写像.
			\item ある任意の集合についてその集合からその集合についての写像.
		\end{enumerate}
	\item
		\begin{enumerate}
			\item 高知工科大学の情報学群の学生の中で数学1を履修している生徒の集合について、先生についての集合においてその授業を担当するような先生の集合への写像.
			\item ある高知県に住んでいる人の集合について住民のから高知県の市区町村の集合への写像.
		\end{enumerate}

	\item $\forall b \exists a ( b = f(a))$

	\item
		\begin{enumerate}
			\item 人間の集合について生物学的に男もしくは女の性別への集合への写像.
			\item 実数全体の集合$x$を定義域として、$f(x) = x ^ 2$を満たすような函数について$x ^2 \in \mathbb{R}$を値域とした集合のへの写像.
		\end{enumerate}
	\item 
		\begin{enumerate}
			\item $f(x)=2x=y$について$x \in \mathbb{N}$、$y \in \mathbb{R}$における函数の関係の写像.
			\item 日本の都道府県の集合について、すべての都道府県に対応する県庁所在地の集合への写像.
		\end{enumerate}

\end{enumerate}

\section{問3}
		\begin{enumerate}
			\item
			\[
                                \chi_A(x) = \left\{
                                \begin{array}{ll}
                                    1 & \text{if } x < 0  \\
                                    0  & \text{otherwise}
                                \end{array}
                                \right.
            \]
			
			集合$U \in \mathbb{Z}$、集合$A \in \mathbb{Z_-}$とする.

			\item
			    \begin{enumerate}
				    \item 
				        $\chi_A(-1) = 1$
			        \item
				        $\chi_A(0) = 0$
			        \item
				        $\chi_A(100) = 0$
				\end{enumerate}
		\end{enumerate}

\section{問4}
	\begin{enumerate}
		\item $r:\mathbb{Z}_+ \rightarrow \mathbb{Z}_+$について、$\forall y \exists x ( r (x) = y )$ただし、$x \in \mathbb{Z}_+, y \in \mathbb{Z}_+$.そのとき、$r(x) = b$を満たす$x$を考える.ここでは、$ a = 3b$とする.そうすると、$3b \mod 3 = 0 $より、$r(a) = r(3b) = \frac{3b}{3} = b$が成り立つ.このことと全射の定義より、$r$は全射である.
		\item $r:\mathbb{Z}_+ \rightarrow \mathbb{Z}_+$について$\forall x_1 \forall x_2 (r(x_1) = r(x_2) \land x_1 = x_2 )$.$x_1 = 1, x_2 = 3$とおく、こととき、$r(1) = 1^2 = 1, r(3) = \frac{3}{3} = 1$となる.よって、$r(x_1) = r(x_2)$が成り立つ.このとき、$1 \neq 3$であることに加えて、$ x_1 \neq x_2 $であるから$r$は単射ではない.
	\end{enumerate}

\section{問5}
	\begin{enumerate}
		\item  $f:X \rightarrow Y$と$g:Y \rightarrow Z$について$X=\{a,b,c,d\}$, $Y=\{1,2,3\}$, $Z=\{z,x,y\}$のとき、写像$f=\{(a,3),(b,1),(c,3),(d,2)\}$, 写像$g=\{(1,x),(2,z),(3,y)\}$として、$g \circ f$という合成写像は$g \circ f = \{(a,y),(b,x),(c,y),(d,z)\}$となる.
	\end{enumerate}

\section{問6}
	\begin{enumerate}
		\item
			$\forall a_1 ,\forall a_2 ( g \circ f(a_1) = g \circ f(a_2) \rightarrow a_1 = a_2) \rightarrow \forall b_1 \forall b_2(f(b_1) = f(b_2) \rightarrow b_1 = b_2)$ただし、$a_1, a_2,b_1, b_2 \in X$

			$ g \circ f$が単射であると仮定する.$f(b_1) = f(b_2)$を満たす$X$の要素$b_1, b_2$を任意にとる.このとき仮定より、$f(b_1) = f(b_2) \rightarrow b_1 = b_2$に加えて、$\forall a_1 \forall a_2 (g \circ f(a_1) = g \circ f(a_2) \rightarrow a_1 = a_2)$が成り立つ.これを写像$f$の$a_1, a_2$について考える.$\forall a_1 \forall a_2 (f(a_1) = f(a_2) \rightarrow a_1 = a_2)$となるより、写像$f$は単射である.

		\item  $\forall c_1, \exists a_1(c_1 = g \circ f(a_1)) \rightarrow \forall c_2 \exists b_1 (c_2 = g(b_1))$ただし、$a_1 \in X, b_1 \in Y , c_1 , c_2 \in \mathbb{Z}$である.
				
			$ g \circ f$が全射であると仮定する.$c \in \mathbb{Z}$となるような、$c$を任意にとる.仮定より、$c = g \circ f(a)$を満たす$X$の要素$a$をとれる.$ g \circ f$が全単射であることに加えて、$f:X \rightarrow Y$より、$f(a) = b$となる$Y$の要素$b$が存在する.よって$\forall c \exists b (c = g(b))$が成り立つため、$g$は全射である.

	\end{enumerate}
\end{document}
